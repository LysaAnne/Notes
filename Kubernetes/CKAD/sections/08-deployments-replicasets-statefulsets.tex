\section{Deployments, ReplicaSets og StatefulSets}
Disse typer er de mest anvendte workload types i Kubernetes. Her kommer mere detaljerede forklaringer på de 3 typer.
\subsection{Deployments}

\begin{itemize}
    \item Den mest anvendte workload type i Kubernetes
    \item Vigtigste funktioner:
    \begin{itemize}
        \item Stateless applications
        \item Autmatisk skalering
        \item Rolling updates og rollbacks
    \end{itemize}
    \item Velegnet til \textbf{stateless applications} (applikationer uden vedvarende tilstand)
    \item Alle pods i et Deployment er identiske
    \item Skalering sker ved at ændre antallet af pods
    \item Understøtter \textbf{rolling updates}, hvor pods gradvist udskiftes
    \item Mulighed for \textbf{rollback} til tidligere version ved fejl
    \item Sikrer minimal downtime under opdateringer
    \item Typiske use cases er webservere, API’er og microservices
\end{itemize}

\subsection{ReplicaSets}

\begin{itemize}
    \item Vigtigste funktioner:
    \begin{itemize}
        \item Sikring af et fast antal pods
        \item Ingen rolling updates eller rollbacks
        \item Bliver ofte styret af Deployments
    \end{itemize}
    \item Minder funktionelt om Deployments, men med færre features 
    \item Ingen understøttelse af rolling updates eller rollbacks
    \item Håndterer automatisk genoprettelse af pods ved node-fejl
    \item Bruges oftest indirekte og styres af Deployments
    \item Anvendes sjældent direkte i praksis
\end{itemize}

\subsection{StatefulSets}

\begin{itemize}
    \item Vigtigste funktioner:
    \begin{itemize}
        \item Designet til \textbf{stateful applications} (applikationer med vedvarende data)
        \item Stabile og unikke netværksidentiteter
        \item Persistent storage til hver pod
    \end{itemize}
    \item Hver pod får et unikt og stabilt navn
    \item Netværksidentitet bevares ved genstart eller genskabelse
    \item Persistent storage tildeles individuelt til hver pod
    \item Pods oprettes og nedskaleres i en fast og kontrolleret rækkefølge
    \item Velegnet til databaser, key-value stores og andre stateful services
\end{itemize}

\subsection{Hvornår bruges de forskellige workload types}

\begin{itemize}
    \item Brug \textbf{Deployment} når applikationen er stateless, kræver nem skalering og automatiske opdateringer
    \item Brug \textbf{ReplicaSet} hvis du kun har behov for at sikre et fast antal pods uden avancerede features
    \item Brug \textbf{StatefulSet} hvis applikationen kræver persistent storage og stabile netværksidentiteter
\end{itemize}