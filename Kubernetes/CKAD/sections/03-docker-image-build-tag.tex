\section{Bygning og tagging af Docker images}

\subsection{Formål}
\begin{itemize}
    \item Omdanne en Dockerfile til et brugbart Docker image
    \item Bruge tags til versionsstyring og overblik
    \item Gøre images klar til brug i Kubernetes og andre miljøer
\end{itemize}

\subsection{Bygning af Docker images}
\begin{itemize}
    \item Docker images bygges ud fra en \texttt{Dockerfile}
    \item Den primære kommando er \texttt{docker build}
    \item Kommandoen opretter et image baseret på instruktionerne i Dockerfile
\end{itemize}

\subsection{docker build-kommandoen}
\begin{itemize}
    \item \texttt{-t}: Angiver image-navn og tag
    \item Image-navn: Typisk navnet på applikationen
    \item Tag: Versions- eller identifikationsnavn (fx \texttt{latest}, \texttt{1.0}, \texttt{dev})
    \item Path: Placeringen af Dockerfile
    \item \texttt{.} betyder den nuværende mappe
\end{itemize}

\subsection{Eksempel på build}
\begin{itemize}
    \item \texttt{docker build -t myapp:1.0 .}
    \item Bygger et image kaldet \texttt{myapp} med tag \texttt{1.0}
    \item Dockerfile findes i den nuværende mappe
\end{itemize}

\subsection{Docker build-processen}
\begin{itemize}
    \item Docker læser Dockerfile linje for linje
    \item Hver instruktion opretter et nyt layer
    \item Layers caches for hurtigere genopbygning
    \item Kun ændrede layers genbygges ved næste build
    \item Alle layers samles til ét færdigt image
\end{itemize}

\subsection{Image tagging}
\begin{itemize}
    \item Tags bruges til at identificere og versionere images
    \item Gør det muligt at styre hvilke versioner der deployes
    \item Understøtter rollback til tidligere versioner
\end{itemize}

\subsection{Best practices for tagging}
\begin{itemize}
    \item Brug semantisk versionering (fx \texttt{1.0.0}, \texttt{1.1.0}, \texttt{2.0.0})
    \item Undgå kun at bruge \texttt{latest}
    \item Brug miljø-tags som \texttt{dev}, \texttt{staging}, \texttt{prod}
    \item Brug Git commit hash som tag for sporbarhed
\end{itemize}

\subsection{Push af images til registry}
\begin{itemize}
    \item Images deles via et container registry
    \item Eksempler: Docker Hub, GitHub Container Registry (GHCR)
    \item Kommandoen er \texttt{docker push}
\end{itemize}

\subsection{Push-eksempler}
\begin{itemize}
    \item \texttt{docker push myapp:1.0}
    \item \texttt{docker push ghcr.io/organisation/myapp:1.0}
    \item Registry-URL skal angives ved private registries
\end{itemize}

\subsection{Login til registry}
\begin{itemize}
    \item Du skal være logget ind før push
    \item Brug \texttt{docker login}
    \item Gælder både Docker Hub og private registries
\end{itemize}

\subsection{Versionshåndtering i praksis}
\begin{itemize}
    \item Hver ændring i koden bør give et nyt image
    \item Nyt build → nyt tag → push til registry
    \item Giver overblik over ændringer og deployment-status
    \item Gør fejlfinding og rollback nemmere
\end{itemize}