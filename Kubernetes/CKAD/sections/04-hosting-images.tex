\section{Container registries og hosting af images}

\subsection{Hvad er en container registry}
\begin{itemize}
  \item En \textbf{container registry} er en lagringstjeneste til Docker images
  \item Fungerer som et centralt sted, hvor images kan gemmes og hentes
  \item Bruges af:
  \begin{itemize}
    \item Kubernetes clusters
    \item CI/CD pipelines
    \item Udviklere og teams
  \end{itemize}
  \item Gør det muligt at distribuere applikationer ensartet på tværs af miljøer
\end{itemize}

\subsection{Populære container registries}
\begin{itemize}
  \item \textbf{Docker Hub}
  \begin{itemize}
    \item Standard registry for Docker
    \item Meget udbredt til både public og private images
  \end{itemize}
  \item \textbf{GitHub Container Registry (GHCR)}
  \begin{itemize}
    \item Registry integreret direkte med GitHub
    \item Images gemmes sammen med kildekoden
  \end{itemize}
  \item \textbf{Google Container Registry (GCR)}
  \begin{itemize}
    \item Googles løsning til container images
    \item Bruges ofte sammen med Google Cloud og GKE
  \end{itemize}
  \item \textbf{Amazon Elastic Container Registry (ECR)}
  \begin{itemize}
    \item AWS’ registry til sikker opbevaring af Docker images
  \end{itemize}
  \item \textbf{Harbor}
  \begin{itemize}
    \item Open source registry fra CNCF
    \item Bruges ofte i enterprise- og on-premise miljøer
  \end{itemize}
\end{itemize}

\subsection{Push af images til Docker Hub}
\begin{itemize}
  \item Kræver en Docker Hub konto
  \item Første trin er at logge ind:
  \begin{itemize}
    \item \texttt{docker login}
  \end{itemize}
  \item Image skal tagges med dit Docker Hub brugernavn
  \item Tag-format:
  \begin{itemize}
    \item \texttt{username/myapp:1.0}
  \end{itemize}
  \item Image pushes til registry med:
  \begin{itemize}
    \item \texttt{docker push username/myapp:1.0}
  \end{itemize}
  \item Efter push kan imaget hentes fra enhver maskine med adgang til Docker Hub
\end{itemize}

\subsection{GitHub Container Registry (GHCR)}
\begin{itemize}
  \item Velegnet hvis GitHub allerede bruges til source control
  \item Understøtter sikker lagring og adgangsstyring
  \item Kræver en \textbf{Personal Access Token (PAT)} fra GitHub
  \item Token bruges som password ved login:
  \begin{itemize}
    \item \texttt{docker login ghcr.io}
  \end{itemize}
  \item Image skal tagges med \texttt{ghcr.io} prefix
  \item Efter push gemmes imaget i GitHub og kan bruges af Kubernetes eller CI/CD
\end{itemize}

\subsection{Public vs private repositories}
\begin{itemize}
  \item \textbf{Public repositories}
  \begin{itemize}
    \item Alle kan hente imaget
    \item Velegnet til open source projekter
  \end{itemize}
  \item \textbf{Private repositories}
  \begin{itemize}
    \item Kræver autentificering
    \item Bruges til proprietære eller interne applikationer
  \end{itemize}
  \item Indstillinger styres:
  \begin{itemize}
    \item Via Docker Hub webinterface eller CLI
    \item Via GitHub repository- og package-indstillinger
  \end{itemize}
\end{itemize}

\subsection{Versionering af container images}
\begin{itemize}
  \item Versionering er afgørende for stabilitet og sporbarhed
  \item Brug altid entydige tags (fx \texttt{1.0.0}, \texttt{2.1}, commit-SHA)
  \item Undgå at bruge \texttt{latest} i produktion
  \item Forkert brug af \texttt{latest} kan føre til:
  \begin{itemize}
    \item Uventede opdateringer
    \item Fejl i deployment
  \end{itemize}
\end{itemize}