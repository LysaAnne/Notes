\section{Introduktion til Dockerfile}
\subsection{Hvad er en Docker file}
\begin{itemize}
  \item En \textbf{Dockerfile} er en tekstfil, der beskriver, hvordan et container image bygges
  \item Dockerfile fungerer som en opskrift for Docker
  \item Kvaliteten af Dockerfile påvirker performance, skalerbarhed og sikkerhed
  \item Dårligt skrevne Dockerfiles kan føre til:
  \begin{itemize}
    \item Lange build-tider
    \item Store og ineffektive images
    \item Sikkerhedssårbarheder
  \end{itemize}
\end{itemize}

\subsection{Grundlæggende opbygning af en Dockerfile}
\begin{itemize}
  \item \textbf{FROM}: Angiver base image (fx alpine, node, python)
  \item \textbf{RUN}: Udfører kommandoer under build (installerer software, konfiguration)
  \item \textbf{COPY / ADD}: Kopierer filer fra host til container
  \item \textbf{CMD}: Standardkommando der køres, når containeren starter
  \item \textbf{ENTRYPOINT}: Definerer containerens primære kommando
\end{itemize}

\subsection{Best practices for effektive Dockerfiles}
\subsubsection{Minimal base image}
\begin{itemize}
  \item Brug så små base images som muligt
  \item Undgå fulde OS-images, hvis det ikke er nødvendigt
  \item Foretræk fx \texttt{alpine} eller \texttt{slim}-varianter
  \item Mindre images giver:
  \begin{itemize}
    \item Hurtigere pulls
    \item Mindre lagerforbrug
    \item Mindre angrebsflade
  \end{itemize}
\end{itemize}

\subsubsection{Multi-stage builds}
\begin{itemize}
  \item Brug ét image til at bygge applikationen og et andet til at køre den
  \item Build-værktøjer inkluderes kun i build-staget
  \item Runtime-imaget indeholder kun det nødvendige output
  \item Resultat: mindre og renere images
\end{itemize}

\subsection{Sikkerhed i Dockerfiles}
\subsubsection{Undgå root-brugeren}
\begin{itemize}
  \item Containere kører som root som standard
  \item Dette er en sikkerhedsrisiko
  \item Opret en non-root bruger i containeren
  \item Kør applikationen som denne bruger
  \item Reducerer risikoen for privilege escalation
\end{itemize}

\subsubsection{Minimer installerede pakker}
\begin{itemize}
  \item Hver installeret pakke øger angrebsfladen
  \item Installer kun absolut nødvendige afhængigheder
  \item Hold base image og pakker så simple som muligt
\end{itemize}

\subsubsection{Opdatering og scanning}
\begin{itemize}
  \item Sårbarheder i afhængigheder er almindelige
  \item Opdater base images og pakker regelmæssigt
  \item Brug sikkerhedsscannere:
  \begin{itemize}
    \item Docker Scan
    \item Trivy
  \end{itemize}
\end{itemize}

\subsection{Performance-optimering af Dockerfiles}
\subsubsection{Minimer antal lag}
\begin{itemize}
  \item Hver \texttt{RUN}, \texttt{COPY} og \texttt{ADD} skaber et nyt lag
  \item Mange lag giver større images og langsommere builds
  \item Kombinér kommandoer, hvor det er muligt
\end{itemize}

\subsubsection{Udnyt Docker caching}
\begin{itemize}
  \item Docker cacher hvert lag i Dockerfile
  \item Lag, der ikke ændres, genbruges ved rebuild
  \item Placer sjældent ændrede instruktioner øverst:
  \begin{itemize}
    \item Installation af afhængigheder
    \item Base image
  \end{itemize}
  \item Ofte ændrede filer (fx applikationskode) placeres senere
\end{itemize}

\subsection{Eksempel på et clean og sikkert Dockerfile}
\begin{itemize}
  \item Bruger et minimalt Alpine base image
  \item Anvender multi-stage build
  \item Kører applikationen som non-root bruger
  \item Kopierer kun nødvendige filer til runtime-imaget
  \item Resultat:
  \begin{itemize}
    \item Lille image-størrelse
    \item Hurtig startup
    \item Forbedret sikkerhed
  \end{itemize}
\end{itemize}