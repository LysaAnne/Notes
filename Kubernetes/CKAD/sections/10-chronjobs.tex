\section{CronJobs}

\subsection{Hvad er et CronJob}

\begin{itemize}
    \item En workload type til \textbf{tidsplanlagte} opgaver i Kubernetes
    \item Fungerer på samme måde som cron jobs i Linux
    \item Bruges til at køre opgaver på faste tidspunkter eller intervaller
    \item Et CronJob opretter automatisk et \textbf{Job}, hver gang tidsplanen rammes
    \item Hvert Job kører til opgaven er fuldført og afsluttes derefter
\end{itemize}

\subsection{Hvordan CronJobs fungerer}

\begin{itemize}
    \item CronJob-controlloren overvåger tidsplanen
    \item Når tidsplanen udløses, oprettes et nyt Job
    \item Jobbet starter én eller flere pods, som udfører opgaven
    \item Når opgaven er færdig, markeres Jobbet som completed
\end{itemize}

\subsection{Planlægning og cron-syntaks}

\begin{itemize}
    \item Tidsplanen angives som en cron expression
    \item Samme syntaks som klassisk Linux cron
    \item Felter fra venstre mod højre:
    \begin{itemize}
        \item Minute
        \item Time
        \item Dag i måneden
        \item Måned
        \item Ugedag
    \end{itemize}
    \item Eksempel: kør hver dag kl. 00:00
    \item Eksempel: kør hver mandag kl. 09:00
\end{itemize}

\subsection{Konfiguration af CronJobs}

\begin{itemize}
    \item \textbf{Schedule}: angiver hvornår jobbet skal køre
    \item \textbf{Job template}: beskriver hvilket Job der skal oprettes
    \item \textbf{Concurrency policy}: styrer hvad der sker, hvis et tidligere Job stadig kører
    \item Mulighed for automatisk oprydning af gamle Jobs
    \item Kan begrænse antal gemte succesfulde og fejlede Jobs
\end{itemize}

\subsection{Typiske anvendelser}

\begin{itemize}
    \item Database backups på faste tidspunkter
    \item Log rotation, komprimering eller arkivering
    \item Generering af rapporter og automatiske e-mails
    \item Systemvedligeholdelse og oprydning af midlertidige filer
\end{itemize}

\subsection{Hvornår skal man bruge CronJobs}

\begin{itemize}
    \item Når opgaver skal køre på bestemte tidspunkter
    \item Når opgaver skal gentages regelmæssigt
    \item Når Kubernetes automatisk skal håndtere oprettelse af Jobs
\end{itemize}

\subsection{Hvornår CronJobs ikke er velegnede}

\begin{itemize}
    \item Når opgaver skal køres manuelt eller on-demand
    \item Når opgaver kræver dynamisk horisontal skalering
    \item I disse tilfælde er Jobs, Deployments eller StatefulSets mere passende
\end{itemize}

\subsection{Opsummering}

\begin{itemize}
    \item CronJobs automatiserer gentagne og tidsstyrede opgaver
    \item De reducerer behovet for manuel drift
    \item Et vigtigt værktøj til vedligeholdelse af Kubernetes-clustre
\end{itemize}