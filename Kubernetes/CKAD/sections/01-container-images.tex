\section{Container Images i Kubernetes}

\subsection{Hvad er et container image}
\begin{itemize}
  \item Et \textbf{container image} er en letvægts, portabel pakke, der indeholder alt en applikation skal bruge for at køre.
  \item Indeholder applikationskode, dependencies (afhængigheder), miljøvariabler, konfiguration og nødvendige OS-biblioteker.
  \item Er \textbf{ikke} et fuldt operativsystem, men kun det minimale, der er nødvendigt for at køre applikationen.
  \item Sikrer at applikationen altid kører i det samme, forventede miljø.
\end{itemize}

\subsection{Container images som grundsten i Kubernetes}
\begin{itemize}
  \item Alle containers i Kubernetes startes ud fra et container image.
  \item Gælder for alle typer workloads:
  \begin{itemize}
    \item Webapplikationer
    \item Backend-services
    \item Sidecars
    \item Systemværktøjer (fx metrics exporters)
  \end{itemize}
  \item Image-kvaliteten påvirker hele applikationens livscyklus:
  \begin{itemize}
    \item Opstartstid
    \item Hukommelsesforbrug
    \item Sikkerhed
    \item Stabilitet
  \end{itemize}
\end{itemize}

\subsection{Deklarativt setup med images}
\begin{itemize}
  \item Kubernetes er \textbf{deklarativt}.
  \item Du beskriver \textbf{hvad} der skal køre, ikke \textbf{hvordan}.
  \item I et YAML-manifest definerer du:
  \begin{itemize}
    \item Hvilket image der skal bruges
    \item Hvilken version (tag)
    \item Hvor mange replicas der skal køre
  \end{itemize}
  \item Kubernetes sørger selv for at:
  \begin{itemize}
    \item Hente imaget fra et registry
    \item Starte containers som specificeret
  \end{itemize}
\end{itemize}

\subsection{Image versionering og tags}
\begin{itemize}
  \item Image-tags bruges til at versionere applikationer.
  \item Brug af \texttt{latest} anbefales ikke:
  \begin{itemize}
    \item Giver uforudsigelige deployments
    \item Gør fejlsøgning og rollback sværere
  \end{itemize}
  \item En ny deployment kræver ofte kun:
  \begin{itemize}
    \item Ændring af image-tag i YAML-filen
  \end{itemize}
  \item Korrekt versionering er afgørende for \textbf{reproducerbare builds}.
\end{itemize}

\subsection{Immutable Infrastructure}
\begin{itemize}
  \item Kubernetes arbejder med \textbf{immutable infrastructure}.
  \item Ved fejl i produktion:
  \begin{itemize}
    \item Ruller man tilbage til et tidligere image
    \item Ikke til tidligere kode direkte
  \end{itemize}
  \item Rollback gendanner både:
  \begin{itemize}
    \item Applikationen
    \item Dens miljø
  \end{itemize}
  \item Dette øger stabilitet og robusthed i systemet.
\end{itemize}

\subsection{OCI-standarden}
\begin{itemize}
  \item Kubernetes bruger ikke sit eget image-format.
  \item Følger \textbf{OCI-standarden (Open Container Initiative)}.
  \item Images kan bygges med:
  \begin{itemize}
    \item Docker
    \item Podman
    \item Buildah
  \end{itemize}
  \item Images gemmes i et \textbf{container registry}, fx:
  \begin{itemize}
    \item Docker Hub
    \item GitHub Container Registry
    \item Private registries
  \end{itemize}
  \item Kubernetes kræver kun:
  \begin{itemize}
    \item Adgang til registry
    \item Korrekt image-path og tag
  \end{itemize}
\end{itemize}